% Options for packages loaded elsewhere
\PassOptionsToPackage{unicode}{hyperref}
\PassOptionsToPackage{hyphens}{url}
%
\documentclass[
  11pt,
  ignorenonframetext,
]{beamer}
\usepackage{pgfpages}
\setbeamertemplate{caption}[numbered]
\setbeamertemplate{caption label separator}{: }
\setbeamercolor{caption name}{fg=normal text.fg}
\beamertemplatenavigationsymbolsempty
% Prevent slide breaks in the middle of a paragraph
\widowpenalties 1 10000
\raggedbottom
\setbeamertemplate{part page}{
  \centering
  \begin{beamercolorbox}[sep=16pt,center]{part title}
    \usebeamerfont{part title}\insertpart\par
  \end{beamercolorbox}
}
\setbeamertemplate{section page}{
  \centering
  \begin{beamercolorbox}[sep=12pt,center]{part title}
    \usebeamerfont{section title}\insertsection\par
  \end{beamercolorbox}
}
\setbeamertemplate{subsection page}{
  \centering
  \begin{beamercolorbox}[sep=8pt,center]{part title}
    \usebeamerfont{subsection title}\insertsubsection\par
  \end{beamercolorbox}
}
\AtBeginPart{
  \frame{\partpage}
}
\AtBeginSection{
  \ifbibliography
  \else
    \frame{\sectionpage}
  \fi
}
\AtBeginSubsection{
  \frame{\subsectionpage}
}
\usepackage{amsmath,amssymb}
\usepackage{lmodern}
\usepackage{iftex}
\ifPDFTeX
  \usepackage[T1]{fontenc}
  \usepackage[utf8]{inputenc}
  \usepackage{textcomp} % provide euro and other symbols
\else % if luatex or xetex
  \usepackage{unicode-math}
  \defaultfontfeatures{Scale=MatchLowercase}
  \defaultfontfeatures[\rmfamily]{Ligatures=TeX,Scale=1}
\fi
\usetheme[]{metropolis}
% Use upquote if available, for straight quotes in verbatim environments
\IfFileExists{upquote.sty}{\usepackage{upquote}}{}
\IfFileExists{microtype.sty}{% use microtype if available
  \usepackage[]{microtype}
  \UseMicrotypeSet[protrusion]{basicmath} % disable protrusion for tt fonts
}{}
\makeatletter
\@ifundefined{KOMAClassName}{% if non-KOMA class
  \IfFileExists{parskip.sty}{%
    \usepackage{parskip}
  }{% else
    \setlength{\parindent}{0pt}
    \setlength{\parskip}{6pt plus 2pt minus 1pt}}
}{% if KOMA class
  \KOMAoptions{parskip=half}}
\makeatother
\usepackage{xcolor}
\newif\ifbibliography
\usepackage{graphicx}
\makeatletter
\def\maxwidth{\ifdim\Gin@nat@width>\linewidth\linewidth\else\Gin@nat@width\fi}
\def\maxheight{\ifdim\Gin@nat@height>\textheight\textheight\else\Gin@nat@height\fi}
\makeatother
% Scale images if necessary, so that they will not overflow the page
% margins by default, and it is still possible to overwrite the defaults
% using explicit options in \includegraphics[width, height, ...]{}
\setkeys{Gin}{width=\maxwidth,height=\maxheight,keepaspectratio}
% Set default figure placement to htbp
\makeatletter
\def\fps@figure{htbp}
\makeatother
\setlength{\emergencystretch}{3em} % prevent overfull lines
\providecommand{\tightlist}{%
  \setlength{\itemsep}{0pt}\setlength{\parskip}{0pt}}
\setcounter{secnumdepth}{-\maxdimen} % remove section numbering
\ifLuaTeX
  \usepackage{selnolig}  % disable illegal ligatures
\fi
\IfFileExists{bookmark.sty}{\usepackage{bookmark}}{\usepackage{hyperref}}
\IfFileExists{xurl.sty}{\usepackage{xurl}}{} % add URL line breaks if available
\urlstyle{same} % disable monospaced font for URLs
\hypersetup{
  pdftitle={Análisis de presencias con procesos de puntos},
  pdfauthor={Gerardo Martín},
  hidelinks,
  pdfcreator={LaTeX via pandoc}}

\title{Análisis de presencias con procesos de puntos}
\subtitle{Particularidades}
\author{Gerardo Martín}
\date{2022-06-29}

\begin{document}
\frame{\titlepage}

\begin{frame}{Supuestos}
\protect\hypertarget{supuestos}{}
\emph{Todos hacemos suposiciones y casi todas estan mal} (Einstein)

\begin{itemize}
\tightlist
\item
  Identificar bajo qué condiciones podemos estar equivocadxs
\end{itemize}
\end{frame}

\begin{frame}{Supuestos}
\protect\hypertarget{supuestos-1}{}
\textbf{Estadísticos} - Supuestos \(\rightarrow\) Errores potenciales
\(\rightarrow\) Soluciones potenciales

\textbf{Biológicos} - Supuestos estadísticos \(\rightarrow\) Problema de
estudio \(\rightarrow\) Interpretaciones
\end{frame}

\begin{frame}{Supuestos estadísticos}
\protect\hypertarget{supuestos-estaduxedsticos}{}
\begin{itemize}
\item
  Variable analizada / Modelo estadístico
\item
  Significado de los resultados
\item
  MPPs \(\rightarrow\) diferentes supuestos estadísticos

  \begin{itemize}
  \tightlist
  \item
    Distribución estadística de presencias
  \item
    Independencia
  \item
    Sesgo observacional
  \end{itemize}
\end{itemize}
\end{frame}

\begin{frame}{Supuestos estadísticos - Ejemplos}
\protect\hypertarget{supuestos-estaduxedsticos---ejemplos}{}
Media aritmética

\begin{itemize}
\tightlist
\item
  Valor más probable en distribución normal
\end{itemize}

\includegraphics{Particularidades_files/figure-beamer/unnamed-chunk-1-1.pdf}
\end{frame}

\begin{frame}{Supuestos estadísticos - Ejemplos}
\protect\hypertarget{supuestos-estaduxedsticos---ejemplos-1}{}
\includegraphics{Particularidades_files/figure-beamer/unnamed-chunk-2-1.pdf}
\end{frame}

\begin{frame}{Supuestos de MPPs}
\protect\hypertarget{supuestos-de-mpps}{}
\begin{itemize}
\tightlist
\item
  Intensidad de puntos promedio (\(\lambda(u)\)) tiene distribución
  Poisson
\item
  Los puntos son \textbf{independientes}
\item
  \(\lambda(u)\) es log-lineal
\end{itemize}
\end{frame}

\hypertarget{dependencia-espacial}{%
\section{Dependencia espacial}\label{dependencia-espacial}}

\begin{frame}{Autocorrelación}
\protect\hypertarget{autocorrelaciuxf3n}{}
Puntos se repelen \(\rightarrow\) Puntos son independientes
\(\rightarrow\) Puntos se atraen

\begin{center}\includegraphics{Figuras/Ejemplo-procesos} \end{center}
\end{frame}

\begin{frame}{Autocorrelación}
\protect\hypertarget{autocorrelaciuxf3n-1}{}
Moran-\emph{I} \textgreater{} 1, valores similares son cercanos:

\begin{center}\includegraphics{Figuras/Moran-1-1} \end{center}
\end{frame}

\begin{frame}{Autocorrelación}
\protect\hypertarget{autocorrelaciuxf3n-2}{}
Moran-\emph{I} \(\approx\) 0, no hay ningún patrón

\begin{center}\includegraphics{Figuras/Moran-2-1} \end{center}
\end{frame}

\begin{frame}{Supuestos biologicos}
\protect\hypertarget{supuestos-biologicos}{}
\begin{itemize}
\item
  Dependen de estadisticos

  \begin{itemize}
  \tightlist
  \item
    Si
  \item
    No
  \end{itemize}
\end{itemize}
\end{frame}

\end{document}
